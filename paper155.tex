\documentclass[runningheads,a4paper]{llncs}

%
\usepackage{amssymb}
\usepackage{amsmath}
\usepackage{url}
\usepackage{graphicx}
\usepackage{multirow}
%
\newcommand{\argmin}[1]{\underset{#1}{\operatorname{arg}\operatorname{min}}\;}
\mathchardef\mhyphen="2D
\raggedbottom 
%
% Allow easy processing of labeled images in figures
\newcounter{lfigcounter}
\newenvironment{IonTab}{\begin{table}[htb]}{\end{table}}
\newenvironment{IonFig}{\setcounter{lfigcounter}{1}\begin{figure}} {\end{figure}}
\newenvironment{IonFigH}{\setcounter{lfigcounter}{1}\begin{figure}[h]}{\end{figure}}
\newenvironment{IonFigT}{\setcounter{lfigcounter}{1}\begin{figure}[!t]}{\end{figure}}
\newenvironment{IonFigB}{\setcounter{lfigcounter}{1}\begin{figure}[b]}{\end{figure}}
\def\ionbox#1{\makebox[#1]{(\alph{lfigcounter})}\stepcounter{lfigcounter}}
%
\begin{document}

\mainmatter  % start of an individual contribution

% first the title is needed
\title{An Image-based instantaneous phase estimation method for gating and temporal super-resolution of cardiac ultrasound}

% a short form should be given in case it is too long for the running head
\titlerunning{Image-based Phase Estimation Method for Cardiac Ultrasound}

% the name(s) of the author(s) follow(s) next
%
% NB: Chinese authors should write their first names(s) in front of
% their surnames. This ensures that the names appear correctly in
% the running heads and the author index.
%

% anonymous stuff
\author{*}
\authorrunning{*}   
\tocauthor{*}
\institute{*}

\maketitle

\begin{abstract}
Point-of-care low-cost hand-held ultrasound devices are emerging as an increasingly effective tool for rapid non-invasive visual assessment of cardiac structure and function on the bed-side. Their diagnostic value can be enriched further by coupling them with automated image analysis algorithms to accurately detect and quantify any cardiac abnormalities at the point-of-care potentially even in asymptomatic patients. However, a major challenge to this is the poor spatio-temporal resolution and high noise level of images obtained using these devices. In an effort to address this problem, we present a novel method for de-noising, gating, and temporal super-resolution of noisy low frame-rate periodic image sequences such as the ones obtained using low-cost hand-held cardiac ultrasound devices. We first leverage the inherent image-level redundancy in a periodic image sequence to decouple signal from noise using a low-rank + sparse decomposition algorithm. Next, we transform the complex high-dimensional periodic image sequence to a narrow-band 1D time series with the same periodicity characteristics using a novel strategy based on inter-frame similarity. Next, we decompose this time series use a season-trend decomposition method to obtain the heart-beat (season) and respiratory (trend) signals. Next, we use the Hilbert transform to estimate the instantaneous cardiac and respiratory phases of each frame to facilitate gating and super-resolution. Lastly, we use Nadarya-Watson kernel regression to construct an image manifold parameterized by intra-period phase which can then be sampled to reconstruct a de-noised single-period image sequence at a higher temporal-resolution. We demonstrate the utility and efficacy of our method both on synthetic data and on real image sequences of the beating heart acquired using a low-cost hand-held ultrasound device.
%\keywords{Cardiac, Ultrasound, Phase estimation, Gating, Temporal super-resolution}
\end{abstract}

\vspace{-0.5cm}
\section{Introduction}
\label{sec:intro}
%

%
\vspace{-0.5cm}
\section{Method}
\label{sec:method}
%

%
\vspace{-0.5cm}
\subsection{Instantaneous phase estimation of each frame}
\label{sec:method:phase_estimation}
%

%
\vspace{-0.5cm}
\subsection{Gating of respiratory frames}
\label{sec:method:gating}
%

%
\vspace{-0.5cm}
\subsection{Reconstruction of one-period video at high temporal resolution}
\label{sec:method:gating}
%

\vspace{-0.5cm}
\section{Results}
\label{sec:results}
%

%
\vspace{-0.5cm}
\section{Discussion and future work}
\label{sec:discussion}
%


%
\vspace{-0.5cm}
\bibliographystyle{splncs03}
\bibliography{library}
\end{document}







